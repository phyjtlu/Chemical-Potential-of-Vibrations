\documentclass[aps,prb,
%,twocolumn
,floatfix,footinbib,longbibliography,
preprint
]{revtex4-1}
%\documentclass[aps,preprint,floatfix,footinbib,longbibliography]{revtex4-1}
\usepackage{epsfig}
\usepackage{graphicx}% Include figure files
\usepackage{dcolumn}% Align table columns on decimal point
\usepackage{bm}% bold math
\usepackage{mathrsfs}
\usepackage{amsmath}
\usepackage{bbold}
\usepackage{color,xcolor}
\usepackage{epstopdf}
\usepackage{subfigure} 
%\newcommand{\equt}[1]{\stackrel{#1}{=}}
%\usepackage[backend=bibtex,sorting=none,style=trad-abbrv,citestyle=numeric]{biblatex}

%\usepackage[sorting=none]{biblatex}
%\usepackage{hyperref}
%\usepackage[titletoc]{appendix}
% avoids incorrect hyphenation, added Nov/08 by SSR
\hyphenation{ALPGEN}
\hyphenation{EVTGEN}
\hyphenation{PYTHIA}

\usepackage[colorlinks=true,pdfborder=001,linkcolor=blue,anchorcolor=blue,citecolor=blue,urlcolor=blue]{hyperref}

\newcommand{\revision}[1]{{\color{blue}{#1}}}



\begin{document}

%\preprint{APS/123-QED}

%\title{Electron-hole pair excitation and energy transport in hybrid electron-boson junctions}
%\title{Electron-hole pair excitation and unified description of hybrid energy transport in electron-boson nano-junctions}

%\title{Charge transfer electron-hole pair excitation and energy transport in hybrid electron-boson nano-junctions}

\title{Nonequilibrium reservoir engineering of a biased coherent conductor for hybrid energy transport in nanojunctions}

% \title{Hybrid electron-boson system as a electron-hole pair exciton reservoir}

\author{Bing-Zhong Hu}
%\author{Tao Wang}
\author{Lei-Lei Nian}
\email{llnian@hust.edu.cn}
\author{Jing-Tao L\"{u}}
\email{jtlu@hust.edu.cn}
\affiliation{School of Physics and Wuhan National High Magnetic Field Center, Huazhong University of Science and Technology, Wuhan 430074, P. R. China}


\date{23 March, 2019}% It is always \today, today,%but any date may be explicitly specified
\begin{abstract}
%We propose a Landauer-B\"uttiker formula to describe  energy transport between weakly coupled hybrid electron-boson nano-junctions. The electron bath is described by effective bosonic baths with possibly nonzero chemical potential.  Our theory gives a unified account of current-induced heating/cooling, electroluminescence and thermoelectric transport in nano-junctions. Our results extend the Landauer formalism to hybrid electron-boson systems and shed light on the nature of energy transfer between electron and boson systems.

%We show that a current-carrying coherent electron conductor can be treated as effective bosonic energy reservoir involving different types of electron-hole pair excitation. Hybrid energy transport between nonequilibrium electrons and bosons can be described by a Landauer-B\"uttiker formula. This allows for intuitive and unified account of a variety of heat transport problems in hybrid electron-boson systems, including non-reciprocal heat transport between electrons and bosons, thermoelectric current from a cold-spot and radiative cooling. 
We show that a current-carrying coherent electron conductor can be treated as effective bosonic energy reservoir involving different types of electron-hole pair excitation. For weak electron-boson coupling, hybrid energy transport between nonequilibrium electrons and bosons can be described by a Landauer-like formula. This allows for unified account of a variety of heat transport problems in hybrid electron-boson systems. As applications, we study the non-reciprocal heat transport between electrons and bosons, thermoelectric current from a cold-spot and electronic cooling of the bosons.
Our unified framework provides an intuitive way of understanding hybrid energy transport between electrons and bosons in their weak coupling limit. It opens the way of nonequilibrium reservoir engineering for efficient energy control between different quasi-particles in the nanoscale.


%Our results extend the Landauer formalism to hybrid electron-boson systems and shed light on the nature of energy transfer between electron and boson systems.

%Our theory paves the way of designing hybrid quantum devices for efficient energy control in the nanoscale.
%\begin{description}
%\item[PACS numbers]
%73.63.Kv, 73.23.-b, 71.38.-k, 72.25.-b
%\end{description}
\end{abstract}


\maketitle
%\tableofcontents

%\section{Introduction-1}
%Chemical potential is a key concept in thermodynamics and statistical mechanics. As the conjugated variable to the particle number, chemical potential difference drives particle transport or chemical reactions.

\section{Introduction}
%\emph {Introduction.--} 
%20200223--------------------------------------
%Understanding nonequilibrium energy transport in the nanoscale is of crucial importance both for the fundamental development of quantum thermodynamics and for the practical application of nanoscale thermal, thermoelectric and optoelectronic devices. Energy carriers following different statistics, including electrons\cite{imry1999conductance}, photons\cite{ojanen2008mesoscopic,biehs2010mesoscopic,zhang2018energy,benabdallah2014near}, phonons\cite{rego1998quantized,mingo2005carbon,yamamoto2006nonequilibrium,wang2006nonequilibrium,wang2007nonequilibrium,wang2008quantum,ruokola2009thermal,li2012colloquium,taylor2015quantum,wang2016landauer} and magnons\cite{wang2004spin}, have been used for the control of nanoscale energy transport. Theoretical approaches developed for each quasi-particle can be readily applied in these studies.
%20200223--------------------------------------

Understanding nonequilibrium energy transport in the nanoscale is of crucial importance both for the fundamental development of quantum thermodynamics and for the practical application of nanoscale thermal, thermoelectric and optoelectronic devices. For phase coherent transport, the celebrated Landauer-B\"uttiker formalism has been successfully applied to study quasi-particle energy transport following different statistics, including electrons\cite{imry1999conductance}, photons\cite{ojanen2008mesoscopic,biehs2010mesoscopic,zhang2018energy,benabdallah2014near}, phonons\cite{rego1998quantized,mingo2005carbon,yamamoto2006nonequilibrium,wang2006nonequilibrium,wang2007nonequilibrium,wang2008quantum,ruokola2009thermal,li2012colloquium,taylor2015quantum,wang2016landauer} and magnons\cite{wang2004spin}.
%The energy current between two baths ($i=1,2$) can be written as the following general form
%\begin{equation}
%J_{}=\int_{0}^{+\infty}\frac{d\omega}{2\pi}\hbar\omega T(\omega)[n_{}(\omega,\mu_{1},T_{1})-n_{}(\omega,\mu_{2},T_{2})].
%\label{eq:lb}
%\end{equation}
Wherein, the baths connecting to the system are assumed to be in thermal equilibrium with given temperature and/or chemical potential, where the quasi-particle distribution function  is determined by its statistics, i.e., the Fermi-Dirac distribution for fermions, and the Bose-Einstein distribution for bosons. A difference in the distribution drives an energy current flow between the two thermal baths. 

%This driving force for energy transport could be a chemical potential or temperature bias for fermions. However, for bosons we have $\mu=0$ in thermal equilibrium, following the textbook argument that bosons without number conservation have zero chemical potential. Thus, for bosonic energy transport the only driving force is temperature.


%Here, the transmission coefficient $T(\omega)$ describes the transmission probability of particle with energy $\hbar\omega$. The distribution function $n(\omega,\mu,T)$ is determined by the particle statistics. It is the Fermi-Dirac distribution for fermions. Either a chemical potential or a temperature bias can drive an energy current flow. Thus, thermoelectric transport can also be studied using this equation. For bosons, $n$ is the Bose-Einstein distribution. Equation~\ref{eq:lb} has been used to study temperature-driven bosonic energy transport carried by phonons, photons and other quasi-particles. Here, we have $\mu=0$, following the textbook argument that bosons without number conservation have zero chemical potential. Thus, for bosonic energy transport the only driving force is temperature.

However, the same approach is difficult to describe energy transport between quasi-particles following different statistics, which is ubiquitous in thermoelectric and optoelectronic processes of nano-junctions. Examples of such processes include electroluminescence\cite{kuhnke2017atomic,galeprin2017photonics,schneider2010optical,schneider2012light}, Joule heating\cite{huang2007local,ioffe2008detection,lu_current-induced_2015,hartle2011resonant,hartle2011vibrational,hartle2018cooling}, current-induced\cite{galperin2009cooling,simine2012vibrational,lykkebo2016single,hartle2011resonant} or radiative cooling\cite{zhu2019near}. Another difficulty arising in these processes is that the quasi-particles may be in nonequilibrium state due to driving from external bias.

In this work, we show that these processes can be conveniently analyzed by treating a voltage-biased coherent electron conductor as effective bosonic reservoirs with non-zero chemical potentials. In the limit of weak electron-boson coupling, to the second order in their coupling, we obtain a Landauer formula to describe energy transport between electrons and bosons. This is possible since energy transport between electrons and bosons is always accompanied by the generation or annhilation of different kinds of electron-hole pairs (EHPs)\cite{headgordon_molecular_1995,dou2018perspective}. This provides a simple physical picture to understand qualitatively energy transport in different thermal, thermoelectric and optoelectronic processes.  

%However, the generation and annihilation of bosons may be accompanied by transitions between different states of particles with non-zero chemical potentials. This is certainly the case for electrically driven processes. In such situations, it is known that the chemical potential of bosons does not have to be zero. In this work, we show that, energy transport between steady-state nonequilibrium electrons and bosons can be well described by a Landauer-B\"uttiker formula between bosonic baths with different chemical potentials. The key observation is that, the electronic system can be treated as bosonic electron-hole-pair bath with a non-zero chemical potential. This general formula can give a unified account of different physical processes involving energy transfer between fermionic and bosonic systems, including current-induced heating/cooling, electrically-driven light emission and thermoelectric transport in nano-scale junctions.







%\begin{align}
%n_{\alpha}(\omega,\nu_{\alpha},T_{\alpha})=\frac{1}{e^{\beta_\alpha(\hbar\omega-\mu_{\alpha})}-1}
%\end{align}
% is the Bose-Einstein
%distribution function in bath $\alpha (=1, 2)$ with the inverse temperature $\beta_\alpha=(k_BT_{\alpha})^{-1}$, $k_B$ being the Boltzmann constant. We have included the chemical potential $\mu_{\alpha}$, which is normally set to zero for phonon or photon transport.

\section{Theory}
\subsection{System setup}
%\subsection{Model}
%\emph{Model.--} 
We consider a model system schematically shown in Fig.~\ref{fig:ehp} (a). The \emph{system} composed of an independent set of bosonic degrees of freedom (DOF) taken as a set of harmonic oscillators. It couples to two kinds of baths. One is an equilibrium boson bath (ph-bath), modeled by an infinite number of harmonic oscillators (bosonic modes). The other is an electron bath (e-bath), which itself includes a central part ($C$) and two electrodes ($L$ and $R$).  The e-bath may be driven into a nonequilibrium steady state by a voltage bias applied between the two electrodes. We consider the situation where transport in the e-bath can be treated using the coherent Landauer-B\"uttiker picture. Without loss of generality, we assume that the system bosons couple only to the central region of the e-bath. Energy transport between the e-bath and ph-bath takes place through their simultaneous coupling to the system bosons. 
We limit ourselves to non-interacting electrons and weak electron-boson interaction such that a lowest order expansion is valid\cite{paulsson05modeling}. %Extension to interacting electrons is possible\cite{dou_born-oppenheimer_2017}. 
The electrons couple to the `displacement' of the system harmonic oscillators
\begin{equation}
H_{es} = \sum_{i,j,k} M^{k}_{ij}c^\dagger_i c_j u_k.
\label{eq:eboson}
\end{equation}
Here, $M^k_{ij}$ describes the coupling of the system mode $k$ to the electronic transition between states $i$ and $j$, and $u_k$ is the `displacement' operator of the system mode $k$. For phonons, it is the displacement, while for photons it is the vector potential. The system-ph-bath coupling is linear between harmonic oscillators and can be treated exactly.

%The above model is quite general. 

%To be more specific, we consider a molecular conductor as an example. of the e-bath. The system is then made of harmonic vibrations of the molecule. The electrode phonons can serve as the ph-bath. Thus, the model introduced here can be used to study energy transport in molecular conductors\cite{lu2007coupled,lu2011laserlike,lu2016electron}.

%Applications of this model include current-induced heating or electroluminescence. The bosonic system represents molecular vibrations and cavity photon modes, respectively. 


\subsection{Electron-hole pair excitation}
%\emph{ Electron-hole pair excitation.--}
Our key observation is that the energy transport between the system and the e-bath can be modeled by different kinds of reactions between EHPs in the e-bath and the bosonic modes in the system. The creation (annihilation) of the bosonic mode is always accompanied by the inelastic electronic transitions from a filled high (low) energy state to an empty low (hight) energy state. These processes can be considered as recombination and creation of EHPs that have the same energy as the bosonic mode, which we take as positive. They can be expressed in the form of reactions
\begin{align}
e_\alpha + h_\beta \rightleftharpoons b_n,
\label{eq:reaction}
\end{align}
where $e_\alpha$, $h_\beta$ and $b_n$ represent electron in electrode $\alpha$, hole in electrode $\beta$ and bosonic mode $b_n$ of the system. Equivalently, we can write 
\begin{align}
e_\alpha \rightleftharpoons e_\beta + b_n,
\label{eq:reaction2}
\end{align}
representing inelastic electronic transition from electrode $\alpha$ to $\beta$, accompanied by emission of bosonic mode $n$ (forward process). The backward direction corresponds to absorption process.


There are four types of EHPs which we label by the spatial location of the electron ($\alpha$) and hole  ($\beta$) states. They are schematically shown in Fig.~\ref{fig:ehp} (c) and (d) for recombination and creation processes, respectively. They are denoted by EHP-$i$ ($i=1,2,3,4$) and are further divided into two groups. The intra-electrode type includes $1/LL$ and $2/RR$, and inter-electrode type includes $3/RL$, $4/LR$.  Additional to energy transfer between e-bath and system, the generation and recombination of inter-electrode EHPs also involves charge transport across the system. As mentioned, we take the energy of mode  and the EHPs to be positive. 


A generalized detailed balance relation applies to each of reactions
\begin{align}
\frac{\tau _{\alpha\rightharpoonup\beta}}{\tau_{\alpha\leftharpoondown\beta}} = {\rm exp}\left[-\beta_B(\hbar\Omega-\mu_{\alpha\beta})\right].
\label{eq:db}
\end{align}
Here, $\tau_{\alpha\rightharpoonup\beta}$ and $\tau_{\alpha\leftharpoondown\beta}$ are the reaction rates for the forward (boson emission) and backward (boson absorption) processes in Eq.~(\ref{eq:reaction}), respectively. They are obtained from the Fermi golden rule
\begin{align}
\tau_{\alpha\rightharpoonup\beta} &= \frac{2\pi}{\hbar}\sum_{i\in\alpha,f\in\beta}|M^m_{ij}|^2  \delta(\varepsilon_i-\varepsilon_f-\hbar\Omega)  \nonumber\\
&\times n_F(\varepsilon_i-\mu_\alpha)(1-n_F(\varepsilon_f-\mu_\beta)).
\end{align}
Here, $n_{F/B}(\varepsilon,T) = \left[{\rm exp}\left(\beta_B \varepsilon\right)\pm 1\right]^{-1}$ is the Fermi-Dirac/Bose-Einstein distribution, with $\beta_B=(k_BT)^{-1}$,  $\mu_{\alpha\beta}=\mu_\alpha-\mu_\beta$, and $M^m_{ij}=\langle \psi_{i}(\varepsilon_i)|M^{}|\psi_{f}(\varepsilon_f)\rangle$ is the transition matrix element from initial state $i$ in electrode $\alpha$ to final state $f$ in electrode $\beta$. The reverse rate $\tau_{\alpha\leftharpoondown\beta}$ can be written similarly.
Thus, when reaching equilibrium with the EHP bath $\alpha\beta$, the bosonic mode follows a Bose-Einstein distribution at temperature $T_e$ and chemical potential $\mu_{\alpha\beta}$. For intra-electrode processes, $\mu_{\alpha\beta}=0$, we have the normal detailed balance relation, while for inter-electrode processes $\mu_{\alpha\beta}$ is determined by the applied voltage bias. Thus, the bosonic mode may acquire a non-zero chemical potential in nonequilibrium. This is consistent with the equilibrium condition for reaction \ref{eq:reaction}.


%Non-zero chemical potential of the EHPs indicates that energy and heat are not equivalent anymore for the inter-electrode EHPs. Taking the process 4 in Fig.~\ref{fig:ehp}(c) as an example, energy conservation requires $\varepsilon_\alpha = \varepsilon_\beta +\hbar\omega_n$. Due to the finite chemical potential, the heat flowing with the boson is $Q=\hbar\omega_n-\mu_{LR}$.  Similarly, the heat flowing with bosons emitted by process 3 is $Q=\hbar\omega_n-\mu_{RL}$. For $T_e=T_{ph}$ and $\mu_{LR}>0$, we arrive at a seemingly surprising result: the total energy $\hbar\omega_n$ emitted in process 4, normally termed Joule heat, actually include some chemical work. Even more counter-intuitively, when $\mu_{LR}>\hbar\omega_n$, $Q$ becomes negative, meaning that the emitted boson carries negative entropy, which is a resource for work extraction.

\begin{figure}
	%\includegraphics[scale=0.45,angle=0]{schematics-v3.pdf}		
	\centering
	\subfigure[]{
	\includegraphics[width=0.2\textwidth,angle=0]{figures/fig_1a.pdf}
	%\caption{fig1}
	}
	\quad
	\subfigure[]{
	\includegraphics[width=0.2\textwidth,angle=0]{figures/fig_1b.pdf}
	}
	\quad
	\subfigure[]{
	\includegraphics[width=0.2\textwidth,angle=0]{figures-0317/fig_1c.pdf}
	}
	\quad
	\subfigure[]{
	\includegraphics[width=0.2\textwidth,angle=0]{figures-0317/fig_1d.pdf}
	}
	\caption{(a) Schematics of the model we consider. The system consists a set of independent bosonic modes. It couples to an electron bath (e-bath), which is modeled as a conductor including a left (L) and a right (R) electrode, with temperature $T_e$ and chemical potential $\mu_L$ and $\mu_R$, respectively. The system further couples to an external thermal bath (ph-bath) at temperature $T_{ph}$. (b) The electron bath can be treated as four different kinds of electron-hole pair (EHP) baths (1-4), shown in (c). (c-d) Four kinds of EHP recombination (c) and excitation (d) processes. The EHPs are classified according to the spatial location of the electron ($e_\alpha$) and the hole ($h_\beta$).}
	\label{fig:ehp}
\end{figure}




%\emph{Power spectrum and energy transport.--} 
The key quantity to describe the EHP baths is the coupling-weighted power spectrum. It can be written as
\begin{align}
\tilde{\Pi}_{mn}^{\alpha\beta}(\omega) &= \left[n_B(\hbar\omega-\mu_{\alpha\beta},T_e)+\frac{1}{2}\right]\Lambda_{mn}^{\alpha\beta}(\omega).
\label{eq:ppehp}
\end{align}
We have introduced the coupling-weighted EHP density of states (DOS)\cite{lu_current-induced_2012,lu2016electron}
\begin{align}
\Lambda_{mn}^{\alpha\beta}(\omega) &= -\sum_{i\in\alpha,f\in\beta}M^{m}_{fi}M^n_{if}  \delta(\varepsilon_i-\varepsilon_f-\hbar\omega)\nonumber\\
&\times (n_F(\varepsilon_\alpha-\mu_\alpha,T_\alpha)-n_F(\varepsilon_\beta-\mu_\beta,T_\beta))\nonumber\\
&=-\int \frac{d\varepsilon}{2\pi} {\rm tr}[M^m A_\alpha(\varepsilon) M^n A_\beta(\varepsilon-\hbar\omega)] \nonumber\\
&\times (n_F(\varepsilon-\mu_\alpha,T_\alpha)-n_F(\varepsilon-\hbar\omega-\mu_\beta,T_\beta)),
\label{eq:gamma}
\end{align}
which also characterizes the system dissipation due to coupling to the e-bath\cite{lu_current-induced_2012}. Here, $A_\alpha$ is electrode spectrum functional.
Equation~(\ref{eq:ppehp}) follows a form of the fluctuation-dissipation relation for an equilibrium ph-bath, albeit with a possibly non-zero chemical potential  $\mu_{\alpha\beta}$. The intra-electrode EHPs (i=1,2) are always in equilibrium with $\mu_{\alpha\alpha}=0$ and temperature $T_e$. But the two inter-electrode EHPs (i=3, 4) have opposite chemical potential $\mu_{RL}=-\mu_{LR}$. They are non-zero when there is a voltage bias applied.  To this end, we have shown that the nonequilibrium e-bath can be divided into four EHP baths with different chemical potentials.
This effective model is shown in Fig.~\ref{fig:ehp} (b).

%\emph{Detailed balance and effective temperature.--}
%We now proceed to show that, a slightly modified detailed balance relation applies to each of the EHP baths. 
%To simplify the analysis, we consider one bosonic mode with angular frequency $\Omega$. A simple rate equation for the mode population $N$ can be established by considering the forward and backward reaction processes
%\begin{align}
%\dot{N} = B (N+1) - A N,
%\end{align}
%where $B$ and $A$ are the reaction rates for the emission and absorption of bosonic quanta, respectively. They can be calculated by summing over the individual rates $B=\sum_{\alpha\beta}B_{\alpha\beta}$, $A=\sum_{\alpha\beta}A_{\alpha\beta}$.
%, with
%\begin{align}
%B_{\alpha\beta} &= \frac{2\pi}{\hbar}\sum_{i_\alpha,f_\beta}|\langle \psi_{i}(\varepsilon_i)|M^{m}|\psi_{f}(\varepsilon_f)\rangle|^2 \\
%&\times %n_F(\varepsilon_i)(1-n_F(\varepsilon_f))\delta(\varepsilon_i-\varepsilon_f-\hbar\Omega),
%\end{align}
%and $A_{\alpha\beta}$ is obtained by the replacement $\hbar\Omega \to -\hbar\Omega$. As a result, the ratio $A/B$ follows
%The reaction rates of each EHP bath follows a generalized detailed balance relation
%\begin{align}
%\frac{A_{\alpha\beta}}{B_{\alpha\beta}} = {\rm exp}(\beta_B(\hbar\Omega-\mu_{\alpha\beta})).
%\end{align}
%with a possibly nonzero chemical potential $\mu_{\alpha\beta}=\mu_\alpha-\mu_\beta$, as required by the equilibrium condition for reaction \ref{eq:reaction}.

%This means each type of the reaction drives the mode into a Bose-Einstein with a chemical potential $\mu_{\alpha\beta}$. This coincides with the chemical potential of the corresponding EHPs, as required by the equilibrium condition of \ref{eq:reaction}:
%\begin{align}
%\mu_\alpha-\mu_\beta = \mu_p.
%\end{align}
%Here, we have used the fact that chemical potential of holes is the opposite to that of electrons.

%The bosonic mode reaches steady state when $\dot{N}=0$, with
%\begin{align}
%N = \frac{1}{A/B-1}.
%\end{align}
%In equilibrium ($\mu_\alpha=\mu_\beta$), we have $A/B={\rm exp}(\beta_B\hbar\Omega)$. When there is voltage bias applied, the final distribution can not be written as a simple form. Normally, an effective temperature is defined by assuming $N$ follows the Bose-Einstein distribution with zero chemical potential
%\begin{align}
%k_BT_{eff} = \frac{\hbar\Omega}{{\rm ln}(1+N^{-1})}.
%\end{align}
%According to previous discussion, we can equivalently defined an effective chemical potential by assuming $N$ follows the Bose-Einstein distribution at $T_e$ 
%\begin{align}
%\mu_{eff} = \hbar\Omega - k_BT{\rm ln}(1+N^{-1}).
%\end{align}
%These two effective parameters are related through
%\begin{align}
%T_{eff} = \frac{T_e}{1-\mu_{eff}/(\hbar\Omega)}.
%\label{eq:tmu}
%\end{align}
%Several comments are noteworthy at this point. Firstly, in the presence of voltage bias, if the reverse of process 4 is normally enhanced more than process 3, we have a positive $\mu_{eff}$ and consequently $T_{eff}>T_e$. The result is heating of the bosonic mode. In the limiting case shown in Fig.~(\ref{fig:resonant}) (a), resonant enhancement may lead to the extreme case of $\mu_{eff}=\hbar\Omega$, or $T_{eff}\to +\infty$. This marks the instability of the bosonic mode. This case has been analyzed in details in Ref.~\onlinecite{lu2011laserlike}. This instability means that the harmonic approximation is not applicable any more\cite{nitzan2018kinetic}. In the other limiting case (Fig.~\ref{fig:resonant}(b)), process 3 is resonantly enhanced, resulting in $\mu_{eff}>0$ or $T_{eff}<T_{e}$. In this regime, the voltage bias is used to cool the bosonic mode below $T_e$.

%%\subsection{Steady state mode population}
%%The reaction~\ref{eq:reaction} suggests that, when reaching steady state, the bosonic mode inherits the chemical potential of the EHPs. Thus, the bosonic mode may acquire a non-zero chemical potential. This is best illustrated by performing a mode population analysis. 
%%
%%To simplify the analysis, we consider one bosonic mode with angular frequency $\Omega$. A simple master equation for the mode population $N$ can be established by considering the forward and backward reaction processes
%%\begin{align}
%%	\dot{N} = \sum_{\alpha\beta}\left[\tau_{\alpha\rightharpoonup\beta} (N+1) - \tau_{\alpha\leftharpoondown\beta} N\right].
%%\end{align}
%%%According to Eq.~(\ref{eq:db}), each type of the reaction drives the mode into a Bose-Einstein distribution with a chemical potential $\mu_{\alpha\beta}$. This coincides with the chemical potential of the corresponding EHPs, as required by the equilibrium condition of \ref{eq:reaction}:
%%%\begin{align}
%%%	\mu_\alpha-\mu_\beta = \mu_p.
%%%\end{align}
%%%Here, we have used the fact that chemical potential of holes is the opposite to that of electrons. 
%%The steady state population of mode is obtained by setting $\dot{N}=0$, which is written as
%%\begin{align}
%%	N = \frac{1}{\sum_{\alpha\beta}\tau_{\alpha\leftharpoondown\beta}/\sum_{\alpha\beta}\tau_{\alpha\rightharpoonup\beta}-1}.
%%\end{align}
%%In equilibrium ($\mu_L=\mu_R$), we obtain the standard Bose-Einstein distribution with temperature $T_e$ and zero chemical potential. When there is voltage bias applied ($\mu_L\neq\mu_R$), the final distribution can not be written as a simple form. Normally, an effective temperature $T_{\rm eff}$ is defined by assuming $N$ follows the Bose-Einstein distribution with zero chemical potential
%%\begin{align}
%%	k_BT_{\rm eff} = \frac{\hbar\Omega}{{\rm ln}(1+N^{-1})}.
%%\end{align}
%%According to previous discussion, we can equivalently define an effective chemical potential by assuming $N$ follows the Bose-Einstein distribution at $T_e$ 
%%\begin{align}
%%	\mu_{\rm eff} = \hbar\Omega - k_BT_e{\rm ln}(1+N^{-1}).
%%\end{align}
%%These are two equivalent equivalent ways of characterizing the nonequilibrium steady state of the vibrational mode. The two effective parameters are related via
%%\begin{align}
%%	T_{\rm eff} = \frac{T_e}{1-\mu_{\rm eff}/(\hbar\Omega)}.
%%	\label{eq:tmu}
%%\end{align}
%%Several comments are noteworthy at this point. Firstly, in the presence of voltage bias, if the emission process is enhanced more than the absorption process, we have a negative $\mu_{\rm eff}$ and consequently $T_{\rm eff}>T_e$. The result is heating of the bosonic mode. In the limiting case shown in Fig.~\ref{fig:resonant}(a), resonant enhancement may lead to the extreme case of $\mu_{\rm eff}=\hbar\Omega$, or $T_{\rm eff}\to +\infty$. This marks the instability of the bosonic mode. This case has been analyzed in details in Ref.~\onlinecite{lu2011laserlike}. The instability means that the perturbative analysis is not applicable any more\cite{nitzan2018kinetic}. It can be avoided by introducing additional coupling to the ph-bath. The validity of $T_{\rm eff}$ in this case will be analyzed elsewhere\cite{Wang-preprint}. In the other limiting case (Fig.~\ref{fig:resonant}(b)), the absorption process is resonantly enhanced, resulting in $\mu_{\rm eff}>0$ or $T_{\rm eff}<T_{e}$. In this regime, the voltage bias is used to cool the bosonic mode below $T_e$.

%When $eV=\hbar\Omega$, a laser-like instability occurs\cite{lu2011laserlike}. This indicates the failure of our harmonic lowest order analysis\cite{nitzan2018kinetic}.

%Efficiency of a heat engine using the bononic system as working medium
%\begin{align}
%\eta = 1-\frac{T_L}{T_H}\frac{\hbar\Omega-\mu_{eff}}{\hbar\Omega}>1-\frac{T_L}{T_H}.
%\end{align}
%It seems that the efficiency is larger than the Carnot efficiency between $T_L$ and $T_H$. The reason is that, the bosonic mode has a nonzero chemical potential, the energy input from $T_H$ includes not only heat, but also chemical energy. All the chemical energy can be converted to work.

\subsection{Energy transport}
%\emph{ Energy transport.--}
Within this effective EHP model, hybrid energy transport between  electrons and system bosons can be treated as bosonic transport.
To the lowest order approximation, we arrive at a  Landauer-like formula for the energy and particle transport from e-bath to the system as a summation of contributions from all the EHP baths
\begin{align}
J &= \sum_{\alpha,\beta}\int_0^{+\infty}\frac{d\omega}{2\pi}\hbar\omega\ {\rm Tr}[\Lambda^{\alpha\beta}(\omega)\mathcal{A}_{ph}(\omega)]\nonumber\\
&\times [n_B(\omega-\mu_{\alpha\beta},{T}_{e})-n_B(\omega,T_{ph})].
\label{eq:jjtrans}
\end{align}
%\begin{align}
%I^{ph} &= \sum_{\alpha,\beta}\int_0^{+\infty}\frac{d\omega}{2\pi}\ {\rm Tr}[\Lambda^{\alpha\beta}(\omega)\mathcal{A}_{ph}(\omega)]\nonumber\\
%&\times [n_B(\omega-\mu_{\alpha\beta},{T}_{e})-n_B(\omega,T_{ph})].
%\label{eq:iphtrans}
%\end{align}
Here, $J$ is the energy flux from e-bath to ph-bath. $T_{\rm e}$ and $T_{\rm ph}$ are the temperature of the e-bath and ph-bath, respectively. The trace Tr is over system DOF, with $\mathcal{A}_{\rm ph}$ the spectral function of the system due to coupling to the ph-bath. We can write it in terms of the non-interacting boson Green's function $D^{r/a}$ and self-energy $\Pi_{\rm ph}^{r/a}$ as $\mathcal{A}_{ph}=iD^r(\Pi_{\rm ph}^r-\Pi_{\rm ph}^a)D^a$\cite{lu2016electron}. The summation over $\alpha\beta$ includes contributions from all the four types of EHPs. Each of them contributes to one transport channel. 




%where
%\begin{align}
%\mathcal{T}^{\alpha\beta}(\omega)=
%\end{align}
%is the transmission between the EHP bath $\alpha\beta$ and the ph-bath. 

\begin{figure}[h]
	%\includegraphics[scale=0.5,angle=-0]{heating-cooling-v4.pdf}
	\centering
	\subfigure[]{
		\includegraphics[width=0.2\textwidth,angle=0]{figures/fig_2a.pdf}
		%\caption{fig1}
	}
	\quad
	\subfigure[]{
		\includegraphics[width=0.2\textwidth,angle=0]{figures/fig_2b.pdf}
	}
	\caption{Two limiting cases of nonequilibrium reservoir engineering. In (a), we have a filled electronic level $\varepsilon_L$ that couples to the left electrode with chemical potential $\mu_L$, and an empty level $\varepsilon_R$ that couples to the right electrode with chemical potential $\mu_R$. We have $\mu_L>\mu_R$. Heating of the bosonic mode is due to resonant recombination of inter-electrode EHPs (process 4 in Fig.~\ref{fig:ehp}). In (b), the situation is reversed. The left state $\varepsilon_L$ is empty, while the right state $\varepsilon_R$ is filled. When $\mu_R>\mu_L$, the e-bath can be used to cool the bosonic mode through creation of inter-electrode EHPs (process $4'$ in Fig.~\ref{fig:ehp}). }
	\label{fig:resonant}
\end{figure}

In the following we show several applications of this central result.
To be more specific, we consider a minimum model of the e-bath shown in Fig.~\ref{fig:resonant}. We have two electronic states $1$ and $2$ (on-site energies $\varepsilon_1$ and $\varepsilon_2$) couple to the electrodes $L$ and $R$ with coupling parameter $\gamma_1$ and $\gamma_2$, respectively. 
Electron hopping between the two states is assisted by one bosonic mode, which at the same time couples to a ph-bath with coupling constant $\gamma_{ph}$.

Several comments are noteworthy before presenting the numerical results:
\begin{itemize}
\item We limit ourselves to the case of weak electron-boson coupling such that the lowest order results (Eq.~(\ref{eq:jjtrans})) is valid. There are cases where the lowest order result fails qualitatively, see for example Ref.~\onlinecite{nitzan2018kinetic}. When this is the case, the lowest order result shows some instability, indicating the lowest order approximation is not valid anymore. Thus, we have checked our results against more accurate self-consistent Born approximation (SCBA) (dots in Figs.~\ref{fig:JT}-\ref{fig:cooling})to ensue that we are always in the weak coupling limit.  
\item One related issue is the energy conservation. Equation~\ref{eq:jjtrans} can also be obtained from the Born approximation in the nonequilibrium Green's function method, if one considers energy transfer from the system to the ph-bath and includes the electron `bubble' self-energy  to the phonon Green's function\cite{lu2007coupled}. However, if one considers the energy current from e-bath to the system using Born approximation, the result will be different from that of \ref{eq:jjtrans}. Thus, we conclude the Born approximation does not conserve energy. However, if one always keep only the terms to the  lowest order in the electron-boson coupling (the second order)\cite{paulsson05modeling,lu2016electron}, energy current from the above two points of view is still the same and given by Eq.~(\ref{eq:jjtrans}). In conclusion, the lowest order approximation is energy conserving, although it only works for weak electron-boson coupling. 

\item Equation ~(\ref{eq:jjtrans}) has been derived  before\cite{lu2016electron}. The focus of this work is to provide a physical transparent picture to understand this result, such that the numerical results obtained from it can be understood from the coherent bosonic energy transport point of view. In doing so, it can be applied to understand more cases including optoelectronic processes, heat rectification and others, as we demonstrate in next section.

\end{itemize}
\section{Applications}
%\emph{Non-reciprocal heat transport.--}
\subsection{Non-reciprocal heat transport}
Firstly, we consider the situation where the e-bath and ph-bath are in their own thermal equilibrium at two different temperature $T_e$ and $T_{ph}$. This indicates that  $\mu_{L}=\mu_{R}$ and $T_{L}=T_R=T_e$. 
If we ignore the energy dependence of $A$ in Eq.~(\ref{eq:gamma}),
$\Lambda_{mn}(\omega)=\hbar\omega{\rm tr}[M^m A M^n A]$
with $A=A_L+A_R$. Consequently,  the transmission $\mathcal{T}={\rm Tr}[\Lambda\mathcal{A}_{ph}]$ does not depend on $T_e$. Equation~(\ref{eq:jjtrans}) reduces to the Landauer formula for heat transport between two harmonic thermal baths. Thus, the EHPs behave as linear harmonic oscillator thermal baths with a common temperature $T_e$.


\begin{figure}[]
	% \includegraphics[scale=0.6,angle=0]{JT.png}
	% \includegraphics[scale=0.6,angle=0]{RR.png}
	%\subfigure[]{
	    \includegraphics[width=0.45\textwidth,angle=0]{figures/fig_3a.pdf}
	%}
	%\quad
	%\subfigure[]{
	    \includegraphics[width=0.44\textwidth,angle=0]{figures/fig_3b.pdf}	
	%}
	\caption{Non-reciprocal heat transport in a  double dot junction shown in Fig.~\ref{fig:resonant} (a). (a) Heat current as a function of temperature difference $\Delta T/T_{\rm ph}$ for different chemical potentials. \textcolor{blue}{The lines and dots correspond to results from Eq.~(\ref{eq:jjtrans}) and SCBA, respectively. The same applies to Fig.~\ref{fig:coldspot} and Fig.~\ref{fig:cooling}.} (b) Rectification ratio $\rho=(|J(|\Delta T|)|-|J(-|\Delta T|)|)/(|J(|\Delta T|)|+|J(-|\Delta T|)|)/2$ as a function of $\Delta T/T_{\rm ph}$ for different chemical potentials. We consider only one bosonic mode, whose energy is taken as unit energy. The following parameters are used in the calculation: $\varepsilon_L=0.5$, $\varepsilon_R=-0.5$, $\gamma_L=\gamma_R=0.5,m=0.05$, $\hbar\Omega=1$, $k_B=1$. }
	\label{fig:JT}
	% \revision{BZH: (1)Please shift the absolute energy such that $\varepsilon_L=-\varepsilon_R$. (2) Please mark the positions of $\mu$ in Fig. 2(a)  with short lines using the same color coding. (3) Narrow down the $x$ axis to [-0.5,0.5]. Same for Fig. 4. (1)(2)finished.}
\end{figure}

On the other hand, if we consider the energy dependence of $A(\varepsilon)$, $\Gamma(\omega)$, $\mathcal{T}^{}$ will depend on $T_e$. Energy transport becomes anharmonic. In this case, non-reciprocal energy flow is possible, i.e., $J(\Delta T)\neq J(-\Delta T)$, with $\Delta T=T_e-T_{ph}$. We thus find a necessary condition for non-reciprocal energy transport in a hybrid electron-boson system: the electron DOS in the thermal window near the chemical potential has to be energy dependent\cite{zhang2013thermal,ren2013heat}. For normal metal electrode, the energy scale of electrons is much larger than the thermal energy, leading to a flat DOS. The energy dependence of $A(\varepsilon)$ can be engineered by changing the electronic states of the central part. For example, discrete energy levels of a molecular junction or quantum dot can be used.

\begin{figure}[h]

	\includegraphics[width=0.5\textwidth,angle=0]{figures/fig_extra.pdf}

	\caption{Transmission $\mathcal{T}$ as a function of phonon frequency $\omega$ in different e-bath temperature $T_e$,  $\Delta T / T_{ph}$ under chemical potential $\mu=3$, the other parameters are the same as Fig.~\ref{fig:JT}.   }
	\label{fig:DOS}
\end{figure}

In Fig.~\ref{fig:JT} we have considered a two-dot junction shown in Fig.~\ref{fig:resonant} (a). We set $\mu_L=\mu_R=\mu$ and $T_e \neq T_{ph}$ to consider heat transport. The energy dependent EHP DOS gives rise to temperature dependent transmission function $\mathcal{T}^{}$ and consequently heat rectification. Thus, we can manipulate energy current via engineering the e-bath parameters. This shows the advantage of hybrid energy transport compared to the pure bosonic systems. 



 %\revision{Figure~\ref{fig:resonant} shows two limiting cases. We have two electronic levels which couple to the left and right electrodes, respectively. The left level lies below $\mu_L$ is a $n$ state and the right level lies above $\mu_R$ is a $p$ state. Depending on their relative position, one of the inter-electrode EHPs couples strongest to the bosonic system. This setup has been studied in Ref.~\onlinecite{lu2011laserlike}.}







\begin{figure}[h]
	% \includegraphics[scale=0.6,angle=0]{coldspot-v2.pdf}
	% \includegraphics[scale=0.6,angle=0]{IT.png}


	%\subfigure[]{
	    \includegraphics[width=0.45\textwidth,angle=0]{figures/fig_4a.pdf}
	%}
	%\quad
	%\subfigure[]{
	    \includegraphics[width=0.48\textwidth,angle=0]{figures/fig_4b.pdf}	
	%}
	\caption{Thermoelectric efficiency $\eta$ (a)  and electron particle flux $I^e$ (b) as a function of temperature difference between e-bath and ph-bath $\Delta T/T_{ph}$. The parameters are the same as Fig.~\ref{fig:JT}.   }
	\label{fig:coldspot}
\end{figure}


%\emph{ Electrical current from a cold-spot.--}
% \subsection{Hybrid thermoelectric transport}
% We can also study the thermoelectric transport of the temperature-biased electron-boson junction. When $T_{ph}\neq T_e$, in addition to the heat transport between system and e-bath, an electrical current may also be induced between the two electrodes\cite{entinwuhlman2010three,sanchez2011optimal}. In our EHP picture, this is realized through coupling of the bosonic mode with two inter-electrode EHPs. Since they contribute to two electrical current with opposite directions, in order to get a non-zero electrical current, these two channels should not get canceled. The resonant situation in Fig.~\ref{fig:resonant} can be used to enhance one of the two channels. In Fig.~\ref{fig:coldspot}, we show the thermoelectric current induced by the temperature different $\Delta T$ for different chemical potentials $\mu_L=\mu_R$ in the case of Fig.~\ref{fig:resonant} (a). The current is the largest when the chemical potential is in between $\varepsilon_L$ and $\varepsilon_R$, where the resonant enhancement is the most prominent.

% Previously, electrical current generated from a phonon hot-spot ($T_{ph}>T_e$) has been considered\cite{entinwuhlman2010three}. Our results show that the opposite is also possible, where electricity is generated by cooling the ph-bath. This demonstrates the decoupling of heat and charge transport as an advantage of thermoelectricity in hybrid nano-junctions.


\subsection{Hybrid thermoelectric transport}
 We can also study the thermoelectric transport of the temperature-biased electron-boson junction. When $T_{ph}\neq T_e$, in addition to the heat transport between system and e-bath, an electrical current may also be induced between the two electrodes\cite{entinwuhlman2010three,sanchez2011optimal}. In our EHP picture, this is realized through coupling of the bosonic mode with two inter-electrode EHPs. Since they contribute to the electrical current with opposite directions, in order to get a non-zero electrical current, these two channels should not get canceled. We can write the electron particle flux as
 \begin{align}
 I^e &= \sum_{\alpha,\beta} (\delta_{\alpha L}\delta_{\beta R} - \delta_{\alpha R}\delta_{\beta L})   \int_0^{+\infty}\frac{d\omega}{2\pi}\ {\rm Tr}[\Lambda^{\alpha\beta}(\omega)\mathcal{A}_{ph}(\omega)]\nonumber\\
 &\times [n_B(\omega-\mu_{\alpha\beta},{T}_{e})-n_B(\omega,T_{ph})].
 \label{eq:ietrans}
 \end{align}
 Here, $\delta_{\alpha/\beta, L/R}$  are the Kronecker delta functions. For simplicity, we introduce thermoelectric efficiency $\eta$ as the ratio between electron particle flux and phonon particle flux $\eta= I^e / (J^{ph}/\hbar\Omega)$.

% Obviously, all 4 EHP baths contribute phonon particle flux
% \begin{align}
% I^{ph} &= I^{ph}_{LL} + I^{ph}_{LR} + I^{ph}_{RL} + I^{ph}_{RR}.
% \label{eq:iph}
% \end{align}

% Here, $I^{ph}_{\alpha\beta}$ is the phonon particle flux contribution of EHP bath $\alpha\beta$

% \begin{align}
% I^{ph}_{\alpha\beta} &= \int_0^{+\infty}\frac{d\omega}{2\pi}\ {\rm Tr}[\Lambda^{\alpha\beta}(\omega)\mathcal{A}_{ph}(\omega)]\nonumber\\
% &\times [n_B(\omega-\mu_{\alpha\beta},{T}_{e})-n_B(\omega,T_{ph})].
% \label{eq:iphab}
% \end{align}

% On the other hand, only LR and RL baths contribute electron particle flux. Further more, according to EHP reaction equation equation (\ref{eq:reaction}) and equation (\ref{eq:reaction2}), 
% \begin{align}
% I^e_{LR}=I^{ph}_{LR},~I^e_{RL}=-I^{ph}_{RL}. 
% \label{eq:ie}
% \end{align}


% Thus, electron particle flux can be expressed by phonon particle flux contribution
% \begin{align}
% I^{e} =I^{e}_{LR} + I^{e}_{RL} = I^{ph}_{LR} - I^{ph}_{RL}.
% \label{eq:ie}
% \end{align}

% Finally, our thermoelectric current generation efficiency 
% \begin{align}
% \eta = \frac{I^{ph}_{LR} - I^{ph}_{RL}}{I^{ph}_{LL} + I^{ph}_{LR} + I^{ph}_{RL} + I^{ph}_{RR}}.
% \label{eq:tcge}
% \end{align}


 The resonant situation in Fig.~\ref{fig:resonant} can be used to enhance one of the two channels. In Fig.~\ref{fig:coldspot} (b), we show the thermoelectric current induced by the temperature different $\Delta T$ for different chemical potentials $\mu_L=\mu_R$ in the case of Fig.~\ref{fig:resonant} (a). The efficiency $\eta$ in Fig.~\ref{fig:coldspot} (a) is the largest when the chemical potential is in between $\varepsilon_L$ and $\varepsilon_R$, where the resonant enhancement is the most prominent.
 Previously, electrical current generated from a phonon hot-spot ($T_{ph}>T_e$) has been considered\cite{entinwuhlman2010three}. Our results show that the opposite ($T_{ph}<T_e$) is also possible, where electricity is generated by cooling the ph-bath.  
 
 Although it sounds a bit counter intuitive, we note that generating electricity by cooling the bosonic bath does not violate the second law of thermodynamics. The energy flows from the high temperature e-bath to the low temperature ph-bath in accordance with the second law. The energy source comes from the recombination of the inter-electrode EHPs. During this process, electronic charge is transferred from one electrode to the other. Indeed, this kind of setup demonstrates the decoupling of heat and charge transport direction as an advantage of thermoelectricity in hybrid nano-junctions. 

%Notably, when $T_{ph}<T_{e}$ and $\mu_L=\mu_R$. The temperature difference between e-bath and ph-bath generates a heat current flow from the e-bath to the ph-bath. At a result of the heat transport, electron transport between $L$ and $R$ electrode takes place.

%\emph{Electronic cooling of bosonic mode.--}
\subsection{Electronic cooling of bosonic mode}
We now turn on the voltage bias in the e-bath.
The applied voltage changes the initial and final electron states of the EHP excitation. Thus, the EHP DOS can be modified by voltage. More importantly, the inter-electrode EHPs acquire a non-zero chemical potential, EHP-4 has a chemical potential of $\mu_{LR}$, while EHP-3 gets a chemical potential with opposite value $\mu_{RL}$. 
\begin{figure}[h]
	% \includegraphics[scale=0.55,angle=0]{Cooling.png}
	\includegraphics[width=0.5\textwidth,angle=0]{figures/fig_5.pdf}	
	\caption{Energy current $J$ from the e-bath to the bosonic mode as a function of chemical potential $\mu_{LR}$, corresponding to the situation in Fig.~\ref{fig:resonant} (b). Negative $J$ (gray shaded area) means cooling of the bosonic mode.  }
	\label{fig:cooling}
\end{figure}
Change of the chemical potential breaks the equilibrium in the reaction, and drives the energy transport between e-bath and the system. 
Direction of energy flow depends on the relative magnitude of two fluxes. It can be engineered by tuning the electronic band structure, or more specifically, the transition probability of the two types of EHP excitation. 

In the case shown in Fig.~\ref{fig:resonant} (b), process $4'$ is resonantly enhanced. The EHP corresponds to this dominant process has a negative effective chemical potential $\mu_{RL}<0$, lower than that of the ph-bath, which is always zero.  This unbalance in the chemical potential drives an energy currrent from the ph-bath to the e-bath. Electronic cooling becomes possible using this resonant enhancement. This is demonstrated in Fig.~\ref{fig:cooling}, where the heat current from the e-bath to the system $J$ is plotted as a function of voltage bias $\mu_{LR}$ while keeping temperature fixed $T_e=T_{ph}=T$. For negative bias, we observe a negative $J$ regime. The range of this regime gets larger for higher temperature $T$. This is the electronic cooling of the bosonic mode. Very recently, experimental demonstration of near field radiative cooling using a reversely biased $p$-$n$ junction has been demonstrated \cite{zhu2019near}. The experimental results can be understood using this simple model.



%Joule heating in molecular conductors has been studied 

%The forward reaction $e_L+h_R \to p_n$  is enhanced, leading to energy flow from the EHPs to the bosonic mode. Meanwhile, that of $e_R+h_L \to p_n$ is reduced, leading to an opposite energy flow. Direction of the total energy current is determined by the relative magnitude of the two processes, i.e., the magnitude of the transmission $T^{\alpha\beta}$. As an illustration, we consider electronic cooling of the bosonic mode. 
%\begin{figure}[]
%	\includegraphics[scale=0.5,angle=0]{ep.png}
%	\caption{Emergence of exceptional point due to coupling to a nonequilibrium e-bath. The effective dynamical matrix is given in Eq.~(\ref{eq:kk}). The following parameters are used: $\Omega=1$, $\delta=0.04$, $\gamma=0.001$, $a=0.05$, $b=0.0$, $c=0.02$. }
%	\label{fig:ep}
%\end{figure}

%\emph{Current-induced exceptional point in two-mode system.--}
%\subsection{Current-induced exceptional point in two-mode system}
%So far, we have only considered one bosonic mode in the system. Now we show that the nonequilibrium e-bath can be used to couple two otherwise isolated bosonic modes. The bias dependence of coupling parameters can be used to tune the system to an exceptional point, where both the eigen values and eigen vectors of the two modes coalesce. 
%
%To illustrate this effect, we consider two identical bosonic modes with average frequency $\Omega$ and a small detuning $\delta$, such that $\omega_{\pm}=\Omega\pm \delta/2$. Extra damping of the two modes $\gamma_1$ and $\gamma_2$ are introduced to account for their coupling to e-bath and ph-bath. Importantly, the nonequilibrium nature of the e-bath introduces coherent coupling between the two otherwise isolated modes, which are proportional to the applied bias $V$. Putting together, we have the following effective dynamical matrix for the two modes
%\begin{equation}\label{eq:kk}
%\left[
%\begin{array}{cc}
%(\Omega+\delta/2+i \gamma)^2 & V(a+i b \Omega) + i c\Omega \\
%-V (a+i b \Omega)+i c\Omega & (\Omega-\delta/2+i \gamma)^2
%\end{array}
%\right].
%\end{equation}
%Here, $a$ and $b$ are non-zero only in the presence of voltage bias in the e-bath, which correspond to the currend-induced non-conservative and effective Lorentz force, respectively\cite{lu_blowing_2010}. Meanwhile, $c$ and $\gamma$ account for the damping due to coupling to ph-bath and/or e-bath. If we ignore the energy dependence of electron spectral function within the voltage bias window, $a={\rm Im}{\rm tr}[M^1 A_L M^2 A_R]/\pi$ becomes constant, $b=0$. This approximation is valid if the electron bandwidth $D$ is much larger than the corresponding mode energy $\hbar\Omega$, i.e., $D\gg \hbar\Omega$. Figure~\ref{fig:ep} shows a typical example for this model. We have plotted the imaginary (a), the real (b) part of the eigen values as a function of bias. The exceptional point corresponds to the place where both parts are the same for the two modes. Additionally, at this point the inner product of the two eigen vectors takes the maximum value 1, meaning that the eigen vectors coalesce.


\section{Conclusions}
In summary, we have shown that a normal two-probe coherent electron conductor can be effectively viewed as EHP baths with chemical potential determined by the applied voltage bias. This is made possible by introducing the inter-electrode charge transfer EHPs. Properties of the EHP baths can be engineered through tuning the parameters of the conductor and the external voltage bias. This bath engineering provides an efficient way of controlling hybrid energy and thermoelectric transport in electron-boson junctions.  Moreover, this point of view can help one to understand qualitatively numerical results from more accurate calculations, i.e. SCBA, althoug it is quantitatively only applicable in the weak electron-boson interaction regime.

% \revision{BZH: Please check the references and use abbreviations for all the journal names.}

\begin{acknowledgements}
The authors thank J.-S. Wang and M. Brandbyge for discussions. This work is supported by the National Key Research and Development Program of China
(Grant No. 2017YFA0403501), the National Natural Science Foundation of China
(Grant No. 21873033) and the program for HUST academic frontier youth team.
\end{acknowledgements}
%%%%%%%%%%%%%%%%%%%%%%%%%%%%%%%%%%%%%%%%%%%%%%%%%%%%5

%\twocolumngrid

\bibliography{LB,gle-review}
\bibliographystyle{apsrev4-1}
\end{document}

%
% ****** End of file apstemplate.tex ******

